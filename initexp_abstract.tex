This paper examines how the estimation results for a standard New Keynesian model with constant gain least squares learning is sensitive to the stance taken on agents' beliefs at the beginning of the sample.  The New Keynesian model is estimated under rational expectations and under learning with three different frameworks for how expectations are set at the beginning of the sample.  The results show that initial beliefs can have an impact on the predictions of an estimated model; in fact previous literature has exposed this sensitivity to explain the changing volatility of output and inflation in the post-war United States.  The results indicate statistical evidence for adaptive learning, however the rational expectations framework performs at least as well as the learning frameworks, if not better, in in-sample and out-of-sample forecast error criteria.  Moreover, learning is not found to better explain time varying macroeconomic volatility any better than rational expectations.  Finally, impulse response functions from the estimated models show that the dynamics following a structural shock can depend crucially on how expectations are initialized and what information agents are assumed to have.
